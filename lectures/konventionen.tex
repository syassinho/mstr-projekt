%-----------------------------------------------------------------------------%
%                                                                             %
%                              konventionen.tex                               %
%                                                                             %
%                               Achim Ibenthal                                %
%                                                                             %
% CONTENT:             Darstellungskonventionen Vorlesung                     %
%                                                                             %
% REVISION HISTORY     02-MAR-07: Initial Version                             %
%                                                                             %
%-----------------------------------------------------------------------------%

%-----------------------------------------------------------------------------%
%                                Konventionen                                 %
%-----------------------------------------------------------------------------%

\nosection{Konventionen}
\label{sec:konv}
\scriptonly{\addcontentsline{toc}{section}{Konventionen}}


%-----------------------------------------------------------------------------%


\begin{frame}[fragile]
\frametitle{Texthervorhebungen}

 \begin{tabular}{p{0.25\hsize}p{0.7\hsize}}
   \hline
   \vspace{-0.7ex}\textbf{Stil} 
   & \vspace{-0.7ex}\textbf{Erl�uterung} \\[0.2ex] \hline

   \vspace{-0.7ex}\te{wichtig!} 
   & \vspace{-0.7ex}Hervorhebung wichtiger Begriffe \\[0.5ex]

   \me{Datei $\to$ �ffnen} 
   & Anwendungsmen�s \\[0.5ex]

   \cblack{a = b\symbol{94}2 + c\symbol{94}2;}
   & Matlab Quellcode\\[0.5ex]

   \cblueb{fft2}, \cblueb{semilogy}
   & Keywords, Funktionsnamen \\[0.5ex]

   \cgreen{\% Kommentar} 
   & Kommentar im Quelltext\\[0.3ex]

   \noalign{\hrule} \\
\end{tabular}

\end{frame}


%-----------------------------------------------------------------------------%

\begin{frame}%[allowframebreaks,fragile]
\frametitle{Textboxen}

\ablock{Rot}{
\begin{itemize}
    \item Wichtige Ergebnisse
    \item Zusammenfassungen
\end{itemize}
}


\nblock{Blau}{
\begin{itemize}
   \item Definitionen
   \item S�tze
   \item Beweise und Rechnungen
\end{itemize}
}

\slideonly{
\eblock{Gr�n}{
\begin{itemize}
    \item Beispiele
    \item �bungsaufgaben
\end{itemize}
}
}

\end{frame}


%---------------------------------------EOF-----------------------------------%

