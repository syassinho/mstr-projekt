%-----------------------------------------------------------------------------%
%                                                                             %
%                                readme.tex                                   %
%                                                                             %
%                               Achim Ibenthal                                %
%                                                                             %
% CONTENT:             Blocks and Boxes in beamer using aettings_common.tex   %
%                      and settings_slides.tex                                %
%                                                                             %
% REVISION HISTORY     13-AUG-09: Initial Version                             %
%                                                                             %
%-----------------------------------------------------------------------------%

%\title{Blocks and Boxes using \texttt{settings\symbol{95}common.tex} and 
%       \texttt{settings\symbol{95}slides.tex}}
%\TitleTextBox[Motivation:]{Farbig logisches Boxen Layout.}
%\frame{\titlepage}

\mode<presentation>{

   % Change Title Page Content HERE
   \title[Layout options]{Using 
         \texttt{settings\symbol{95}common.tex}, 
         \texttt{settings\symbol{95}slides.tex},
         \texttt{settings\symbol{95}slides.tex} and
         \texttt{mybeamer.sty}}
   \TitleTextBox[Goal:]{Facilitate the application of the \texttt{beamer} class,
                        harmonize the colors of block environments
                        and amend box environments providing boxes
                        with harmonized colors but without titles.}

   % Insert a title slide
   \frame{\clockinit\titlepage}

}

%-----------------------------------------------------------------------------%
%                            Blocks and Boxes Demo                            %
%-----------------------------------------------------------------------------%

\scrnp

\section{Layout Options}



%-----------------------------------------------------------------------------%
%                                 General Hints                               %
%-----------------------------------------------------------------------------%

\subsection{General Hints}
\label{ssec:genhints}


%-----------------------------------------------------------------------------%

\begin{frame}[allowframebreaks,fragile]
\slideonly{\frametitle{General Hints}}

\begin{itemize}
    \item The following files have been created to enhance the flexibility
          of the \texttt{beamer} class in terms of layout and to provide 
          commands which facilitate its application especially for 
          \te{lectures}:
          \begin{itemize} \normalsize
             \slspace{0.5}
             \item \te{\texttt{settings\symbol{95}common.tex}}: Things
                   which are common to both slide and script.
                   Loads packages, sets layout and theme colours, 
                   adds commands and environments, switches PDF realtime clock
                   for slides on/off, sets author and facility information.
             \slspace{0.5}
             \item \te{\texttt{settings\symbol{95}slides.tex}}: slide-specific
                   layout: defines 35 additional color themes, new
                   logo layout, colour control for theme elements.
             \slspace{0.5}
             \item \te{\texttt{settings\symbol{95}slides.tex}}: script-specific 
                   layout such as page layout and boxes.
             \slspace{0.5}
             \item \te{\texttt{mybeamer.sty}}: commands shortcuts for 
                   standard text colors, beamerclass control, figures,
                   boxes and math. 
          \end{itemize}
    \framebreak
    \item To adapt the \te{overall layout} use Section 2 of     
          \texttt{\symbol{92}format\symbol{92}settings\symbol{95}common.tex}.
    \slspace{1}
    \item To adapt \te{lecture title}, \te{personal data} and \te{legal notice}
          use \te{Section 3} of \scriptonly{\hfill \hfill} \scrlf
          \texttt{\symbol{92}format\symbol{92}settings\symbol{95}common.tex}.
    \slspace{1}
    \item Commands for \te{blocks} and \te{boxes} are explained in \te{this 
          Section}. 
    \slspace{1}
    \item \texttt{\symbol{92}te\{\}} provides \te{text emphasis},
          \texttt{\symbol{92}ce\{\}} \ce{c}haracter \ce{e}mphasis (e.g. for
          glossary) and \texttt{\symbol{92}me\{\}} the marking of
          \me{menue commands}.
          It is recommended to use those because the layout is optimized
          for slides and script individually.
          Defaults are defined in Section 10 of
          \texttt{\symbol{92}format\symbol{92}settings\symbol{95}common.tex}.
          A \texttt{\symbol{92}key\{\}} command is available to add sidenotes
          to the script.
    \slspace{1}
    \item \texttt{beamer} increments counters in 
          \texttt{\symbol{92}section*}, \texttt{\symbol{92}subsection*},
          \texttt{\symbol{92}subsubsection*}. Therefore use new commands
          \texttt{\symbol{92}nosection}, \texttt{\symbol{92}nosubsection},
          \texttt{\symbol{92}nosubsubsection} instead (not starred).
    \framebreak
    \item \te{\texttt{mybeamer.sty}} defines many additional commands, which are
          more or less selfexplanatory when reading \texttt{mybeamer.sty}.
          Examples:
          \begin{itemize} \normalsize
             \slspace{0.5}
             \item Color definitions.
             \slspace{0.5}
             \item commands which operate only in one of the two modes
                   slide and script such as 
                   \texttt{\symbol{92}slideonly\{any text\}} and
                   \texttt{\symbol{92}scriptonly\{any text\}}.
             \slspace{0.5}
             \item e.g. \texttt{\symbol{92}slup\{\#\}} and
                   \texttt{\symbol{92}scrdown\{\#\}} are used for v- 
                   and hspaces.
                   Arguments \texttt{\#} are passed without unit and 
                   interpreted as '\texttt{ex}'.
             \slspace{0.5}
             \item \texttt{\symbol{92}switch\{\#1\}\{\#2\}} separates
                   arguments used in slide mode (\texttt{\#1}) and
                   script mode (\texttt{\#2}).
                   This comes in quite handy e.g. for figure resizing.
             \slspace{0.5}
             \item Often a \texttt{\symbol{92}vspace} is overruled
                   by the \TeX~auto layout and more or less ignored.
                   Here the commands \texttt{\symbol{92}slidebar\{\#\}} and
                   \texttt{\symbol{92}scriptbar\{\#\}} are providing
                   the brute force method to enforce spacing.
             \framebreak
             \item \texttt{\symbol{92}fig} facilitates the insertion
                   of figures:
\begin{verbatim}
		\fig{captiontext}
		    {fig:<label>}
		    {\switch{0.9}{0.7}} % sizes for slide/script
		    {figurename}        % some pdf, jpg or png 
		    \scriptbar{2}       % enforce space in script
\end{verbatim}
             \slspace{0.5}
             \item There are many more, check it out!
         \end{itemize}
\end{itemize}

\end{frame}



%-----------------------------------------------------------------------------%
%                              Blocks and Boxes                               %
%-----------------------------------------------------------------------------%

\subsection{Blocks and Boxes}
\label{ssec:bb}


%-----------------------------------------------------------------------------%

\begin{frame}[fragile]
\frametitle{\texttt{normalblock} and \texttt{normalbox} Environment}

\begin{normalblock}{normalblock Title}
\begin{verbatim}
\begin{normalblock}{normalblock Title}
   ...
\end{normalblock}
\end{verbatim}

\vspace{-3ex}
\begin{itemize}
    \item normalblock item
\end{itemize}

\vspace{-2ex}
\begin{enumerate}
	\item normalblock enumerate
\end{enumerate}
\end{normalblock}

\vspace{5ex}

\begin{normalbox}
\begin{verbatim}
\begin{normalbox}
   ...
\end{normalbox}
\end{verbatim}

\vspace{-3ex}
\begin{itemize}
   \item normalbox item
\end{itemize}

\vspace{-2ex}
\begin{enumerate}
	\item normalbox enumerate
\end{enumerate}
\end{normalbox}

\end{frame}


%-----------------------------------------------------------------------------%

\begin{frame}[fragile]
\frametitle{\texttt{exampleblock} and \texttt{examplebox} Environment}

\begin{exampleblock}{exampleblock Title}
\begin{verbatim}
\begin{exampleblock}{exampleblock Title}
   ...
\end{normalblock}
\end{verbatim}

\vspace{-3ex}
\begin{itemize}
    \item examplelblock item
\end{itemize}

\vspace{-2ex}
\begin{enumerate}
	\item exampleblock enumerate
\end{enumerate}
\end{exampleblock}

\vspace{5ex}

\begin{examplebox}
\begin{verbatim}
\begin{examplebox}
   ...
\end{examplebox}
\end{verbatim}

\vspace{-3ex}
\begin{itemize}
   \item examplebox item
\end{itemize}

\vspace{-2ex}
\begin{enumerate}
	\item examplebox enumerate
\end{enumerate}
\end{examplebox}

\end{frame}


%-----------------------------------------------------------------------------%

\begin{frame}[fragile]
\frametitle{\texttt{alertblock} and \texttt{alertbox} Environment}

\begin{alertblock}{alertblock Title}
\begin{verbatim}
\begin{alertblock}{alertblock Title}
   ...
\end{alertblock}
\end{verbatim}

\vspace{-3ex}
\begin{itemize}
    \item alertblock item
\end{itemize}

\vspace{-2ex}
\begin{enumerate}
	\item alertblock enumerate
\end{enumerate}
\end{alertblock}


\vspace{5ex}

\begin{alertbox}
\begin{verbatim}
\begin{alertbox}
   ...
\end{alertbox}
\end{verbatim}

\vspace{-3ex}
\begin{itemize}
   \item alertbox item
\end{itemize}

\vspace{-2ex}
\begin{enumerate}
	\item alertbox enumerate
\end{enumerate}
\end{alertbox}

\end{frame}



%-----------------------------------------------------------------------------%

\begin{frame}[fragile]
\frametitle{\texttt{block} and \texttt{bodybox} Environment}


The standard \texttt{block} environment defined by \texttt{beamer}
uses theme colors for title, title text, canvas, body text 
and bullets:

\begin{block}{Theme}
\begin{verbatim}
\begin{block}{block Title}
   ...
\end{block}
\end{verbatim}

\vspace{-2ex}
\begin{itemize}
   \item block item
\end{itemize}

\vspace{-2ex}
\begin{enumerate}
	\item block enumerate
\end{enumerate}

\end{block}

We have added a \texttt{bodybox} environment.
It's effect is similar to a \texttt{beamercolorbox} with 
\texttt{block body} color (but vertically more 'tight'):
\begin{bodybox}
\begin{verbatim}
\begin{bodybox}
   ...
\end{bodybox}
\end{verbatim}

\vspace{-3ex}
\begin{itemize}
   \item bodybox item
\end{itemize}

\vspace{-2ex}
\begin{enumerate}
	\item body enumerate
\end{enumerate}
\end{bodybox}


\end{frame}


%-----------------------------------------------------------------------------%

\begin{frame}[fragile]
\frametitle{Shortcuts}

\begin{columns}[c] %[T,t,b totalwidth=<width>]
\begin{column}{0.5\textwidth} %----------- COL 1 -----------%
It turns out that in practice typing and error rate can
be reduced by using \te{shortcuts} for the block and box
environments\switch{:}{. In \texttt{article} mode the 
\blue{\texttt{\symbol{92}nblock}}, \blue{\texttt{\symbol{92}nbox}},
\red{\texttt{\symbol{92}ablock}} and \red{\texttt{\symbol{92}nbox}}
commands can be colored too, using the 
\texttt{\symbol{92}def\symbol{92}scriptboxlayout\{boxed\}} option
in \texttt{settings\symbol{95}common.tex}, Section 2.}
\begin{verbatim}
\nblock{Title}{Body} 
\nbox{Body}
\eblock{Title}{Body}  
\ebox{Body}
\ablock{Title}{Body}
\abox{Body}                              
\tblock{Title}{Body}   
\tbox{Body}
\end{verbatim}
\end{column}                  %-----------------------------%
\begin{column}{0.5\textwidth} %----------- COL 2 -----------%
\textit{Examples:}
\begin{verbatim}
\nblock
   {Problem}
   {Does dark matter smell?}
\end{verbatim}

\slup{2}
\nblock{Problem}{Does dark matter smell?}
\begin{verbatim}
\abox{\centering%
      This is important!}
\end{verbatim}

\slup{3}
\abox{\centering%
      This is important!}
\end{column}                  %-----------------------------%
\end{columns}

\end{frame}


  

%-----------------------------------------------------------------------------%

\begin{frame}[fragile]
\frametitle{\texttt{beamercolorbox} Environment}

The \texttt{beamercolorbox} environment provides boxes which' canvas is derived
from some beamer color.
\te{Text and bullet colors} are derived from the beamer theme colors (\te{!}),
however.
Better use

\nbox{
    \begin{itemize}
	    \item \texttt{\symbol{92}nbox} for blue boxes (n = \ce{n}ormal)
    \end{itemize}
}

\ebox{
    \begin{itemize}
	    \item \texttt{\symbol{92}ebox} for green boxes, also
	          used for \ce{e}xamples
    \end{itemize}
}


\abox{
    \begin{itemize}
	    \item \texttt{\symbol{92}abox} for \ce{a}lert boxes
    \end{itemize}
}

\tbox{
    \begin{itemize}
	    \item \texttt{\symbol{92}tbox} for \ce{t}heme color box,
    \end{itemize}
}
giving you matching bullet colors in presentetion (= beamer) mode
and black bullets in article mode anyway.

\end{frame}


%-----------------------------------------------------------------------------%
%                     Definition and Exercise Environments                    %
%-----------------------------------------------------------------------------%

\subsection{New Environments}
\label{ssec:definitioen}

%-----------------------------------------------------------------------------%

\begin{frame}[fragile,allowframebreaks]
\frametitle{Definitions \texttt{\symbol{92}ndef}}

The beamer \texttt{definition} and \texttt{Definition} environments
are using template colors for boxes, too. 
In \texttt{settings\symbol{95}common.tex} a \te{new environment for definitions}
is created which uses \blue{blue} boxes.
This is recommended, since otherwise the audience has to adapt to different
box colors in case you are just changing the Theme color.

\begin{verbatim}
	\ndef{def:<label>}{Analogk�se}{
	    Das willst Du nicht wirklich wissen!
	\filledend}
\end{verbatim}

\ndef{def:<label>}{Analogk�se}{
    Das willst Du nicht wirklich wissen!
\filledend}


\framebreak %-----------------------------------------------------------------%

\begin{itemize} \normalsize
    \slspace{1}
    \item We don't provide numbers in presentation (= beamer) mode because
          \begin{itemize} \normalsize
             \slspace{0.5}
             \item they don't mean much to the audience 
             \slspace{0.5}
             \item and because they are causing problems when using the beamer 
                   overlay specifications such as \texttt{\symbol{92}uncover<>}.
          \end{itemize}
    \slspace{1}
    \item Numbering in script is local to Sections. 
    \slspace{1}
    \item Note that we had to come up with something else than 
          \texttt{\symbol{92}definition} or \texttt{\symbol{92}Definition} 
          since those were already taken.
          Thus: \texttt{\symbol{92}ndef}
    \slspace{1}
    \item The '\texttt{n}' indicates that normal block color is being used.
 
\end{itemize}

\end{frame}


%-----------------------------------------------------------------------------%

\begin{frame}
\frametitle{Satz \texttt{\symbol{92}nsatz} and
            Theorem \texttt{\symbol{92}ntheorem}}

\te{\texttt{\symbol{92}nsatz}} (Satz) and \te{\texttt{\symbol{92}ntheorem}}
(Theorem) essentially are working in the same way as \texttt{\symbol{92}ndef}, 
except they are producing blue boxes in the script if script boxes are 
enabled in Section 2 of \texttt{settings\symbol{95}common.tex}.

\sldown{0.5}
\ntheo{theo:theo}
      {\textsc{Berti}'s Sicht}
      {Die Realit�t sieht anders aus als die Wirklichkeit.}

\nsatz{satz:ft}{\textsc{Fourier}-Transform}{

\slup{2}
\be
    \label{eq:ft1} 
    S(\omega) & = & \int\limits_{-\infty}^{\infty} s(t) e^{-j \omega t} dt, \\
              &  \multimapdotbothBvert & \nonumber \\
    \slup{3}
    \label{eq:ft2} 
    s(t)      & = & \frac{1}{2 \pi} 
                    \int\limits_{-\infty}^{\infty} S(\omega) e^{j \omega t}
                    d\omega.
\ee
\slup{2}}

\end{frame}


%-----------------------------------------------------------------------------%

\begin{frame}
\frametitle{Korollar \texttt{\symbol{92}ncor} and
            Lemma \texttt{\symbol{92}ntheorem}}

\te{\texttt{\symbol{92}ncor}} (Korollar)/\te{\texttt{\symbol{92}ncor}} (Corollary)
and \te{\texttt{\symbol{92}nlem}} (Lemma) do not produce boxes in script mode.

\nkor{cor:rabbit}
      {Osterhasenp�dagogik}
      {Der Lehrer versteckt das Wissen - die Sch�ler sollen's suchen.\openend}

\scrdown{2}
\ncor{cor:maturity}
      {Maturity}
      {Growing old is mandatory, growing up is optional.\openend}

\scrdown{2}
\nlem{lem:lem}
     {\textsc{Jordan}'s Lemma}
     {In der Elektrotechnik gibt es Probleme, bei denen komplexe Zahlen so
      n�tzlich sind, wie negative Zahlen f�r Probleme mit Geld.\openend}

\sldown{1}\scrdown{2}
The white boxes ''\hspace{-0.7ex}\checkbox'' in script mode are generated using 
\texttt{\symbol{92}openend}.

\end{frame}



%-----------------------------------------------------------------------------%

\begin{frame}%[allowframebreaks,fragile]
\frametitle{Exercises: \texttt{\symbol{92}Uebung} -- \texttt{\symbol{92}Antwort}
            environment pair}

For the application of the \te{\texttt{\symbol{92}Uebung} -- \texttt{\symbol{92}Antwort} environment pair} for exercises and answers
see the \LaTeX~source code of this template.
\slspace{3}
You can change the environment title and language of headings in the 
environment definitions in Section 7 of \texttt{settings\symbol{95}common.tex}.
\slspace{3}
Numbering in script is local to Sections.

\end{frame}


%-----------------------------------------------------------------------------%
%                            Sourcecode Listings                              %
%-----------------------------------------------------------------------------%

\subsection{Sourcecode Listings}
\label{ssec:sourcecode}

%-----------------------------------------------------------------------------%

\begin{frame}[allowframebreaks,fragile]
\slideonly{\frametitle{Sourcecode Listings}}

Source code listings are inserted by using the \te{\texttt{listings}} 
package.
\slspace{2}
This template is configured for \texttt{Matlab} M-file language.
Use Section 11 of \texttt{settings\symbol{95}common.tex} if you wish to 
change the default.


\framebreak %-----------------------------------------------------------------%

Example:
\slideonly{\footnotesize}
\begin{verbatim}
\begin{lstlisting}[title=\cblueb{averagingdemo.m},
                   label=list:average,
                   firstnumber=1,
                   backgroundcolor=\color{ultralightgray}]
% Demo Planarer Tiefpass

%% Bild lesen
b  = imread('zoneplate.tif'); 

%% Filterung
b5  = imfilter(b, fspecial('average', 5),  'replicate');
b9  = imfilter(b, fspecial('average', 9),  'replicate');
b13 = imfilter(b, fspecial('average', 13), 'replicate');

%% Display
subplot(2,2,1); imshow(b,  []);  title('Original');
subplot(2,2,2); imshow(b5, []);  title('5 x 5 Filter');
subplot(2,2,3); imshow(b9, []);  title('9 x 9 Filter');
subplot(2,2,4); imshow(b13, []); title('13 x 13 Filter');
\end{lstlisting}
\end{verbatim}
\slideonly{\normalsize}


\framebreak %-----------------------------------------------------------------%

\begin{lstlisting}[title=\cblueb{averagingdemo.m},
                   label=list:average,
                   firstnumber=1,
                   backgroundcolor=\color{ultralightgray}]
% Demo Planarer Tiefpass

%% Bild lesen
b  = imread('zoneplate.tif'); 

%% Filterung
b5  = imfilter(b, fspecial('average', 5),  'replicate');
b9  = imfilter(b, fspecial('average', 9),  'replicate');
b13 = imfilter(b, fspecial('average', 13), 'replicate');

%% Display
subplot(2,2,1); imshow(b,  []);  title('Original');
subplot(2,2,2); imshow(b5, []);  title('5 x 5 Filter');
subplot(2,2,3); imshow(b9, []);  title('9 x 9 Filter');
subplot(2,2,4); imshow(b13, []); title('13 x 13 Filter');
\end{lstlisting}

\end{frame}




%---------------------------------------EOF-----------------------------------%
