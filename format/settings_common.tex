%-----------------------------------------------------------------------------%
%                                                                             %
%                             settings_common.tex                             %
%                                                                             %
%                               Achim Ibenthal                                %
%                                                                             %
%                                                                             %
% CONTENT:             Common Package and command definitions for beamer      %
%                      slides and scripts.                                    %
%                                                                             %
%                       1. Common Packages                                    %
%                    -> 2. Global Colors & Settings                           %
%                    -> 3. Author, Date and Title Data                        %
%                       4. Script/Slide-specific Text and Character Emphasis  %
%                       5. New Commands                                       %
%                       6. File Paths                                         %
%                       7. Language Adaptations                               %
%                       8. New Block Environments                             %
%                       9. Uebung - Antwort, Definition, Satz, Theorem,       %
%                          Korollar, Lemma, Beispiel                          %
%                      10. Block & Box Shortcuts                              %
%                      11. Boxes in Itemize Environments                      %
%                      12. Listings Configuration                             %
%                      13. Textpos Settings                                   %
%                                                                             %
% REVISION HISTORY     01-AUG-06: Original Version                            %
%                      14-AUG-08: Enhance flexibility, new block environments %
%                      17-FEB-18: Update signal processing logo, Matlab staff %
%                                 training 2016 style                         %
%                                                                             %
%-----------------------------------------------------------------------------%


%-----------------------------------------------------------------------------%
%                            1. Common Packages                               %
%-----------------------------------------------------------------------------%

\usepackage[T1]{fontenc}
\usepackage[latin2]{inputenc}
\usepackage{fancyhdr, fancybox, fancyvrb}
\newcommand\hmmax{0} % default: 3; avoid "Too many math alphabets" error due to bm
\newcommand\bmmax{0} % default: 4; Avoid "Too many math alphabets" error due to bm
\usepackage{bm, amsmath, txfonts, pxfonts, textcomp, relsize, eurosym}
\usepackage[lined]{algorithm2e}
\usepackage{format/mybeamer}
\usepackage{graphicx}
\usepackage{tikz, pgf, calc}
\usetikzlibrary{shapes}
\usetikzlibrary{arrows}
\usepackage[overlay,absolute]{textpos}
\usepackage{rotating}
\usepackage{multicol}
\usepackage{listings, longtable, colortbl} 
%\usepackage{nonfloat}
\usepackage[ngerman]{babel}
%\usepackage[english]{babel}


%-----------------------------------------------------------------------------%
%                        2. Global Colors & Settings                          %
%-----------------------------------------------------------------------------%

% Slide Color Theme as per settings_slides.tex
% 
% GRAY              RED               GREEN             BLUE/PURPLE
% --------------    --------------    --------------    --------------    
% LightSlateGray    Bordeaux          Khaki             Marine
% DarkIvory         BordeauxGray      LightKhaki        DarkSteelBlue
% HoneyDew          FireBrick         DarkKhaki         LightSteelBlue
% Snow              Sienna            OliveGreen        Azure
% DarkSnow                            OliveGreenGray    MediumPurple
%                                     OliveDrab         Thistle [n]
% BROWN             ORANGE/YELLOW     OliveDrabGray     
% --------------    --------------    CamouflageGreen   
% NavajoWhite       Orange            SeaGreen          OTHER
% BurlyWood         Goldenrod         LintGreen         --------------
% DarkWheat         LightGoldenRod    CornSilk          Ethiopia
% Wheat                                                 Jamaica
\def\slidethemecolor{BordeauxGray}

% Slide Text and Box Colors
\definecolor{NormalColor}  {rgb} {0.00 0.00 0.45}              % normal boxes
\definecolor{ExampleColor} {rgb} {0.00 0.30 0.00}              % example boxes
\definecolor{AlertColor}   {rgb} {0.40 0.00 0.00}              % alerted boxes
\definecolor{TextColor}    {rgb} {0.00 0.00 0.00}              % general text

% Slide Clock (requires \clockinit at beginning of first slide!)
\def\slideclock{noclock}            % <showclock|noclock>

% Script Box Colors (normal and alerted boxes only)
\def\scriptboxlayout{boxed}         % <boxed|notboxed>
\definecolor{ScriptNormalColorFrame}{rgb}  {0.902 0.906 0.969} % normal boxes
\definecolor{ScriptNormalColorFill} {rgb}  {0.902 0.906 0.969}
\definecolor{ScriptAlertColorFill}  {rgb}  {0.992 0.902 0.906} % alerted boxes
\definecolor{ScriptAlertColorFrame} {named}{myred}

% Script and Slide Hyperlink Color
\definecolor{ColHyper}{named}{myblue}                          % SteelBlue2


%-----------------------------------------------------------------------------%
%                       3. Author, Date and Title Data                        %
%-----------------------------------------------------------------------------%

\newcommand{\doctitle}          {RECENT TRENDS IN ARTIFICIAL INTELLIGENCE BASED COMPUTER VISION}
\newcommand{\doclongauthor}     {Prof. Dr.-Ing. Achim Ibenthal}
\newcommand{\docauthor}         {A. Ibenthal}
\newcommand{\docshortauthor}    {A. Ibenthal}
\newcommand{\docphone}          {0551/3705-195}
\newcommand{\docmail}           {ibenthal@hawk.de}
\newcommand{\docskype}          {achim.ibenthal}
\newcommand{\docinstitute}      
   {Hochschule f�r angewandte Wissenschaft und Kunst            \\
    Fakult�t f�r Naturwissenschaften und Technik, G�ttingen}
\newcommand{\docshortinstitute} {HAWK [n]}
\newcommand{\docdate}           {SoSe 2024}
\newcommand{\docconsult}
   {\begin{itemize}
	    \item nach Vereinbarung
    \end{itemize}}
\newcommand{\docnutzung}
   {Die Inhalte dieser Vorlesung entstammen haupts�chlich den Quellen 
    \cite{Bu05}, \cite{We88a}, \cite{We88b}, die als vorlesungsbegleitendes 
    Material empfohlen werden.
    Weitere Quellen sind im Text kenntlich gemacht.
    Dar�ber hinaus gehende Inhalte und Bilder sind 
    \beamercopyright A. Ibenthal.
    
    \sldown{1}
    Die Nutzung dieses Dokumentes dient ausschlie�lich der Lehre und ist 
    den Studierenden der HAWK, Fachhochschule Hildesheim, Holzminden, G�ttingen 
    f�r die Dauer ihrer Immatrikulation gestattet. 
    \te{Die Weitergabe, Ver�ffentlichung oder kommer\-zielle Nutzung 
    -- auch nur auszugsweise -- ist nicht gestattet.}
    
    \sldown{1}
    Hinweise zu Fehlern, fehlenden oder falschen Referenzen, sowie weitere
    Kommentare und Anregungen sind stets willkommen und werden zur Verbesserung
    \switch{des Vorlesungsmaterials}{des Scripts} fortlaufend eingearbeitet. 
    \slideonly{\vfill}}
\newcommand{\docpolicy}
   {A part of this lecture's content is based on original material provided in 
    \cite{Bu05}, \cite{We88a}, \cite{We88b}, 
    which are recommended for further reading.
    Additional references are given throughout the document.
    All other content and images are \beamercopyright A. Ibenthal.
    
    \sldown{1}
    The usage of this document is strictly limited to teaching purposes.
    The permission to own an electronic or printed copy of this document is
    exclusively granted to \te{enrolled} students of the HAWK University of 
    Applied Sciences and Arts Hildesheim, Holzminden, G�ttingen.
    \te{Redistribution, publication or commercial usage of the whole or
    parts of the course material is prohibited.}
    
    \sldown{1}
    Hints on typographical errors, missing or false references,
    as well as any further suggestions or comments are appreciated and 
    will be considered for the continuous improvement of
    \switch{course material}{this document}. 
    \slideonly{\vfill}}


%-----------------------------------------------------------------------------%
%            4. Script/Slide-specific Text and Character Emphasis             %
%-----------------------------------------------------------------------------%

\mode<article>{
   \newcommand{\ce}[1]{\underline{#1}}             % Character Emphasis
   \newcommand{\te}[1]{\textit{#1}}                % Text Emphasis
   \newcommand{\se}[1]{#1}                         % Text Emphasis in slide only
   \newcommand{\me}[1]{\green{\textsf{#1}}}        % Menu Emphasis
   \newcommand{\key}[1]{\te{#1}                    % Text Emphasis and Sidenote
     \marginpar[\raggedleft\scriptsize\blue{\textsf{#1}}]{\raggedright\scriptsize\blue{\textsf{#1}}}}
   \newcommand{\sidenote}[1]{                      % Sidenote only
     \marginpar[\raggedleft\scriptsize\blue{\textsf{#1}}]{\raggedright\scriptsize\blue{\textsf{#1}}}}
}
\mode<presentation>{
   \newcommand{\ce}[1]{\red{#1}}                   % Character Emphasis
   \newcommand{\te}[1]{\red{#1}}                   % Text Emphasis
   \newcommand{\se}[1]{\red{#1}}                  % Text Emphasis in slide only
   \newcommand{\me}[1]{\green{\textsf{#1}}}        % Menu Emphasis
   \newcommand{\key}[1]{\te{#1}}                   % Text Emphasis w/o Sidenote
   \newcommand{\sidenote}[1]{}                     % No Sidenote
}
            

%-----------------------------------------------------------------------------%
%                              5. New Commands                                %
%-----------------------------------------------------------------------------%

% Shortcuts
\newcommand{\Mat}{MATLAB\rights~}
\newcommand{\Matintitle}{MATLAB{\footnotesize\raisebox{0.3ex}{\rights}}~}

% MATLAB Highlighting
\newcommand{\f}                [2] {\cblueb{#1}\cblue{(}\cblack{#2}\cblue{)}}
\newcommand{\mat}              [1] {\cblueb{#1}}            
\newcommand{\matinline}        [1] {\colorbox{ultralightgray}{\cblack{#1}}} 
\newcommand{\matcommand}       [1] {\cblack{\symbol{62}\hspace{-0.4ex}\symbol{62}\hspace{0.55ex}#1}}            
\newcommand{\matcommandinline} [1] {\colorbox{ultralightgray}{\matcommand{#1}}} 
\newcommand{\matdoc}           [1] {\matcommand{\cgreen{doc #1}}}            
\newcommand{\mathelp}          [1] {\matcommand{\cgreen{help #1}}}            
\newcommand{\matcomment}       [1] {\cgreen{\% #1}}            
\newcommand{\driver}           [1] {\cgreen{#1}}            
\newcommand{\var}              [1] {\cblack{#1}}
\newcommand{\fnam}             [1] {\cblackb{#1}}            
\newcommand{\macrointitle}     [1] {\cblack{\symbol{92}}\cblackb{#1}}            
\newcommand{\toolbox}          [1] {\credb{#1}}  
\newcommand{\toolboxintitle}   [1] {\cblackb{#1}} 

% Item Tabulation
\newcommand{\itab}[4]{\item \parbox[t]{#1\hsize}{#2}\parbox[t]{#3\hsize}{#4}}
\newcommand{\atab}[3]{\item \parbox[t]{#1\hsize}{#2}\hspace{2ex}\parbox[t]{0.83\hsize-#1\hsize}{#3}}
\newcommand{\aarr}[3]{\iarr \parbox[t]{#1\hsize}{#2}\hspace{2ex}\parbox[t]{0.83\hsize-#1\hsize}{#3}}  

% Text Tabulation
\newcommand{\ttab}[4]{\parbox[t]{#1\hsize}{#2}\parbox[t]{#3\hsize}{#4}}

% Auxiliary Lengths
\newlength{\mytab}
\newlength{\myitab}

% Table Parbox (#1: Lenth/hsize, #2: Text #3: vstretch/ex 
\newcommand{\tabbox}[3]{\parbox[t]{#1\hsize}{#2 \hfill \myrule{#3}}}

% Table Row Stretching
\newcommand{\flexcell}[1]{\parbox[c][\totalheight+4pt][c]{\hsize}{#1}}

% Function Tables
\newcommand{\lbox}[1]{\parbox[t]{0.2\hsize}{\cblueb{#1}\cblue{()}}}
\newcommand{\rbox}[2]{\parbox[t]{0.7375\hsize}{#1 \hfill \myrule{#2}}}

% Changing Font
\newcommand{\changefont}[3]{
   \fontfamily{#1} 
   \fontseries{#2} 
   \fontshape{#3} 
   \selectfont}
\newcommand{\bsf}[1]{{\slleft{4}\changefont{cmr}{m}{n} #1}}

% Matrices: [] Brackets
\renewcommand{\arr}[1]{\left[\begin{matrix}#1\end{matrix}\right]}

% Matrices in Colortable
\newlength{\mywidth}
\newcommand{\colarr}[1]
   {\settowidth{\mywidth}{$\left[\begin{matrix}#1\end{matrix}\right]$}%
    \addtolength{\mywidth}{1.3pt}%
    \makebox[\mywidth][c]{$
    \hspace{5pt}\begin{matrix}#1\end{matrix}\hspace{5pt}%
    \hspace{-\mywidth}%
    \left[\phantom{\begin{matrix}#1\end{matrix}}\hspace{1.5pt}\right]$}}

% Complex Matrices: Adjust Right Bracket
\newcommand{\carr}[1]{\left[\begin{matrix}#1\end{matrix}\hspace{-9pt}\right]}
\newcommand{\ccolarr}[1]
   {\settowidth{\mywidth}{$\left[\begin{matrix}#1\end{matrix}\right]$}%
    \addtolength{\mywidth}{1.3pt}%
    \makebox[\mywidth][c]{$
    \hspace{5pt}\begin{matrix}#1\end{matrix}\hspace{5pt}%
    \hspace{-\mywidth}%
    \left[\hspace{1pt}\phantom{\begin{matrix}#1\end{matrix}}\hspace{-9pt}\right]$}}
    

%-----------------------------------------------------------------------------%
%                               6. File Paths                                 %
%-----------------------------------------------------------------------------%

\graphicspath{{Pics/}
              {../Pics/}
              {../../Pics/}
              {../../../Pics/}
              {../../../../Pics/}
              {../10_Pics/}
              {../../10_Pics/}
              {../../../10_Pics/}
              {../../../../10_Pics/}}

\def\pics{format}   % for /pgfdeclare (logos)
\def\bib{lecture}   % for Bib files


%-----------------------------------------------------------------------------%
%                          7. Language Adaptations                            %
%-----------------------------------------------------------------------------%

%\usepackage[hypcap=false]{caption} % caption language addaptation %\captionsetup{figurename=Figure}   % figurename (requires caption package)
%\captionsetup{tablename=Table}     % tablename (requires caption package) 

% Language Adaptations
%\def\prefacename{Vorwort}%
%\def\refname{Literatur}% 
%\def\bibname{Literaturverzeichnis}% 
%\def\abstractname{Zusammenfassung}%
%\def\chaptername{Kapitel}%
%\def\appendixname{Anhang}%
%\def\contentsname{Inhaltsverzeichnis}% % oder nur: Inhalt
%\def\listfigurename{Abbildungsverzeichnis}%
%\def\listtablename{Tabellenverzeichnis}%
%\def\indexname{Index}%
%\def\figurename{Abbildung}%
%\def\tablename{Tabelle}%  % oder: Tafel
%\def\partname{Teil}%
%\def\enclname{Anlage(n)}% % oder: Beilage(n)
%\def\ccname{Verteiler}%   % oder: Kopien an
%\def\headtoname{An}%
%\def\pagename{Seite}%
%\def\seename{siehe}%
%\def\alsoname{siehe auch}
%
%\def\beamertemplateboldblocks{\setbeamerfont{block title}{size={},series=\bfseries}\setbeamertemplate{blocks}[default]}
%\def\beamertemplatelargeblocks{\setbeamertemplate{blocks}[default]}
%\def\beamertemplateshadowblocks{\setbeamertemplate{blocks}[rounded][shadow=true]}


%-----------------------------------------------------------------------------%
%                                                                             % 
%                          8. New Block Environments                          %
%                                                                             % 
% normalblock: color can be defined independent of theme (4th block env.      %
%              additional to <block|exampleblock|alertblock>                  %
% normalbox:   normalblock without title (good for continuation of            %
%              normalblock with same itemize colors on next frame)            %
% examplebox:  exampleblock without title (good for continuation of           %
%              exampleblock with same itemize colors on next frame)           %
% alertbox:    alertblock without title (good for continuation of             %
%              alertblock with same itemize colors on next frame)             %
%                                                                             % 
% Example:     \begin{normalblock}{Title}                                     %
% --------        \begin{itemize}                                             % 
%                    \item text                                               % 
%                 \end{itemize}                                               % 
%              \end{normalblock}                                              % 
%                                                                             % 
%              \framebreak                                                    %
%                                                                             % 
%              \begin{normalbox}                                              % 
%                 \begin{itemize}                                             % 
%                    \item text ct'd                                          % 
%                 \end{itemize}                                               % 
%              \end{normalbox}                                                % 
%                                                                             % 
% Note:        Colors are defined in settings_slides.tex                      %
% -----                                                                       % 
%                                                                             % 
%-----------------------------------------------------------------------------%

% normalblock
\defbeamertemplate*{block normal begin}{default}{
	\par\vskip\medskipamount%
	\begin{beamerboxesrounded}[upper=block title normal,lower=block body normal,shadow=true]
 		{\raggedright\usebeamerfont*{block title normal}\insertblocktitle}%
    	\raggedright%
    	\usebeamerfont{block body normal}}%
\defbeamertemplate*{block normal end}{default}{
	\end{beamerboxesrounded}\vskip\smallskipamount}
\newenvironment<>{normalblock}[1]{%
    \scriptonly{\textsl{#1}}
	\begin{actionenv}#2%
	    \def\insertblocktitle{#1}%
	    \par%
	    \mode<presentation>{%\usebeamerfont{block}%
	        \setbeamercolor{local structure}{parent=normalbullet}}%
 	    \usebeamertemplate{block normal begin}}
	    {\par%
	    \usebeamertemplate{block normal end}%
    \end{actionenv}}


% normalbox
\defbeamertemplate*{box normal begin}{default}{
  	\par\vskip\medskipamount%
  	\begin{beamerboxesrounded}[upper=block title normal,lower=block body normal,shadow=true]
    	{\raggedright\usebeamerfont*{block title normal}}%
    	\raggedright%
    	\usebeamerfont{block body normal}}%
\defbeamertemplate*{box normal end}{default}
	{\end{beamerboxesrounded}\vskip\smallskipamount}
\newenvironment<>{normalbox}{%
	\begin{actionenv}#1%
	    \par%
	    \mode<presentation>{%\usebeamerfont{block}%
	        \setbeamercolor{local structure}{parent=normalbullet}}%
 	    \usebeamertemplate{box normal begin}}
	    {\par%
	    \usebeamertemplate{box normal end}%
    \end{actionenv}}


% examplebox
\defbeamertemplate*{box example begin}{default}{
  	\par\vskip\medskipamount%
  	\begin{beamerboxesrounded}[upper=block title example,lower=block body example,shadow=true]
    	{\raggedright\usebeamerfont*{block title example}}%
    	\raggedright%
    	\usebeamerfont{block body example}}%
\defbeamertemplate*{box example end}{default}
	{\end{beamerboxesrounded}\vskip\smallskipamount}
\newenvironment<>{examplebox}{%
	\begin{actionenv}#1%
	    \par%
	    \mode<presentation>{%\usebeamerfont{block}%
	        \setbeamercolor{local structure}{parent=example text}}%
 	    \usebeamertemplate{box example begin}}
	    {\par%
	    \usebeamertemplate{box example end}%
    \end{actionenv}}


% alertbox
\defbeamertemplate*{box alerted begin}{default}{
  	\par\vskip\medskipamount%
  	\begin{beamerboxesrounded}[upper=block title alerted,lower=block body alerted,shadow=true]
    	{\raggedright\usebeamerfont*{block title alerted}}%
    	\raggedright%
    	\usebeamerfont{block body alerted}}%
\defbeamertemplate*{box alerted end}{default}
	{\end{beamerboxesrounded}\vskip\smallskipamount}
\newenvironment<>{alertbox}{%
	\begin{actionenv}#1%
	    \par%
	    \mode<presentation>{%\usebeamerfont{block}%
	        \setbeamercolor{local structure}{parent=alerted text}}%
 	    \usebeamertemplate{box alerted begin}}
	    {\par%
	    \usebeamertemplate{box alerted end}%
    \end{actionenv}}


% bodybox
\defbeamertemplate*{box body begin}{default}{
  	\par\vskip\medskipamount%
  	\begin{beamerboxesrounded}[upper=block title,lower=block body,shadow=true]
    	{\raggedright\usebeamerfont*{block title}}%
    	\raggedright%
    	\usebeamerfont{block body}}%
\defbeamertemplate*{box body end}{default}
	{\end{beamerboxesrounded}\vskip\smallskipamount}
\newenvironment<>{bodybox}{%
	\begin{actionenv}#1%
	    \par%
	    \mode<presentation>{%\usebeamerfont{block}%
	        \setbeamercolor{local structure}{parent=blockbullet}}%
 	    \usebeamertemplate{box body begin}}
	    {\par%
	    \usebeamertemplate{box body end}%
    \end{actionenv}}


%-----------------------------------------------------------------------------%
%                                                                             %
%  9. Uebung - Antwort, Definition, Satz, Theorem, Korollar, Lemma, Beispiel  %
%                                                                             %
% Hinweis: F�r Beispiele sollte das Beamer "Beispiel" Environment genutzt     %
% -------- werden.                                                            %
%                                                                             %
%-----------------------------------------------------------------------------%

% Provide Replacement for \section* \subsection*
% (beamer uses internal counter for referencing sections and increases section
% counter by each \section*) -> other relative counters get puzzled
\newcounter{nosection}       
\setcounter{nosection}{-10000}
\newcounter{nosubsection}    
\setcounter{nosubsection}{-10000}
\newcounter{nosubsubsection} 
\setcounter{nosubsubsection}{-10000}
\newcounter{tmp} 

\mode<article>{
\newcommand{\nosection}[1]      {\section*{#1}}
\newcommand{\nosubsection}[1]   {\subsection*{#1}}
\newcommand{\nosubsubsection}[1]{\subsubsection*{#1}}
}

\mode<presentation>{
\newcommand{\nosection}[1]      {\setcounter{tmp}{\value{section}}%
                                 \stepcounter{nosection}%
                                 \setcounter{section}{\value{nosection}}%
                                 \section*{#1}%
                                 \setcounter{section}{\value{tmp}}}
\newcommand{\nosubsection}[1]   {\setcounter{tmp}{\value{subsection}}%
                                 \stepcounter{nosubsection}%
                                 \setcounter{subsection}{\value{nosubsection}}%
                                 \subsection*{#1}%
                                 \setcounter{subsection}{\value{tmp}}}
\newcommand{\nosubsubsection}[1]{\setcounter{tmp}{\value{subsubsection}}%
                                 \stepcounter{nosubsubsection}%
                                 \setcounter{subsubsection}{\value{nosubsubsection}}%
                                 \subsubsection*{#1}%
                                 \setcounter{subsubsection}{\value{tmp}}}
}

% Uebung - Antwort 
\newcounter{NUe}[section]                 % Each new section sets NUe to 0
\def\theNUe{\thesection.\arabic{NUe}}     % Output of number Section.NUe

\mode<article>{                           % Exercise + Answers Script
\newcommand{\Uebung}[3]                   
   {\refstepcounter{NUe}\label{#1}        % To avoid warnings use hyperref
    \hypertarget{#1}{}                    % package option hypertexnames=false
    \begin{block}                         
       {\normalsize\bfseries\sffamily �bung \theNUe: #2}
       #3
    \end{block}}
\newcommand{\Antwort}[2]
   {\begin{block}{L�sung zu �bung \hyperlink{#1}{\ref*{#1}}}
       #2
    \end{block}}
\newcommand{\Exercise}[3]                   
   {\refstepcounter{NUe}\label{#1}        % To avoid warnings use hyperref
    \hypertarget{#1}{}                    % package option hypertexnames=false
    \begin{block}                         
       {\normalsize\bfseries\sffamily Problem \theNUe: #2}
       #3
    \end{block}}
\newcommand{\Answer}[2]
   {\begin{block}{Solution to Problem \hyperlink{#1}{\ref*{#1}}}
       #2
    \end{block}}
}

\mode<presentation>{                      % Exercise + Answers Slides
\newcounter{Uebung}
\newcommand{\Uebung}[3]
   {\refstepcounter{NUe}\label{#1}        % To avoid warnings use hyperref
    \begin{exampleblock}                  % package option hypertexnames=false
       {�bung \theNUe: #2}
       #3
    \end{exampleblock}}
\newcommand{\Antwort}[2]
   {\begin{block}{L�sung zu �bung \ref{#1}}
       #2
    \end{block}}
\newcommand{\Exercise}[3]
   {\refstepcounter{NUe}\label{#1}        % To avoid warnings use hyperref
    \begin{exampleblock}                  % package option hypertexnames=false
       {Problem \theNUe: #2}
       #3
    \end{exampleblock}}
\newcommand{\Answer}[2]
   {\begin{block}{Solution to Problem \ref{#1}}
       #2
    \end{block}}
}


% Definition
\mode<article>{                           % Definiton Script
\newcounter{NDef}[section]                % Each new section sets NDef to 0
\def\theNDef{\thesection.\arabic{NDef}}   % Output of number Section.NDef

\newcommand{\ndef}[3]                   
   {\refstepcounter{NDef}\label{#1}       % To avoid warnings use hyperref
    \hypertarget{#1}{}                    % package option hypertexnames=false
    \begin{normalblock}                         
       {Definition \theNDef: #2.}
       #3
    \end{normalblock}}
}

\mode<presentation>{                      % Definition Slides
\newcounter{NDef}                         % Normal counter, no reset by section
\def\theNDef{\arabic{NDef}}               % Output of number NDef
\newcommand{\ndef}[3]
   {\only<1>{\refstepcounter{NDef}}\label{#1}  % To avoid warnings use hyperref
    \begin{normalblock}                   % package option hypertexnames=false
       {Definition: #2.}                  % \theNDef
       #3
    \end{normalblock}}
}
%\newcommand{\hidendef}[1]{\uncover<2->{#1}\addtocounter{NDef}{-1}}


% Satz; also see settings_script for boxed definitions
\mode<article>{ 
\newcounter{NSatz}[section]
\def\theNSatz{\thesection.\arabic{NSatz}}

\newcommand{\nsatz}[3]                   
   {\refstepcounter{NSatz}\label{#1}
    \hypertarget{#1}{}
    \begin{normalblock}                         
       {Satz \theNSatz: #2.}
       #3
    \end{normalblock}}
}

\mode<presentation>{    
\newcounter{NSatz}   
\def\theNSatz{\arabic{NSatz}}   
\newcommand{\nsatz}[3]
   {\only<1>{\refstepcounter{NSatz}}\label{#1} 
    \begin{normalblock}
       {Satz: #2.}                        % \theNSatz
       #3
    \end{normalblock}}
}


% Theorem; also see settings_script for boxed definitions
\mode<article>{ 
\newcounter{NTheo}[section]
\def\theNTheo{\thesection.\arabic{NTheo}}

\newcommand{\ntheo}[3]                   
   {\refstepcounter{NTheo}\label{#1}
    \hypertarget{#1}{}
    \begin{normalblock}                         
       {Theorem \theNTheo: #2.}
       #3
    \end{normalblock}}
}

\mode<presentation>{    
\newcounter{NTheo}   
\def\theNTheo{\arabic{NTheo}}   
\newcommand{\ntheo}[3]
   {\only<1>{\refstepcounter{NTheo}}\label{#1} 
    \begin{normalblock}
       {Theorem: #2.}                    % \theNTheo
       #3
    \end{normalblock}}
}


% Lemma
\mode<article>{ 
\newcounter{NLem}[section]
\def\theNLem{\thesection.\arabic{NLem}}

\newcommand{\nlem}[3]                   
   {\refstepcounter{NLem}\label{#1}
    \hypertarget{#1}{}
    \begin{normalblock}                         
       {Lemma \theNLem: #2.}
       #3
    \end{normalblock}}
}

\mode<presentation>{    
\newcounter{NLem}   
\def\theNLem{\arabic{NLem}}   
\newcommand{\nlem}[3]
   {\only<1>{\refstepcounter{NLem}}\label{#1} 
    \begin{normalblock}
       {Lemma: #2.}                       % \theNLem
       #3
    \end{normalblock}}
}

% Korollar // Corollary
\mode<article>{ 
\newcounter{NCor}[section]
\def\theNCor{\thesection.\arabic{NCor}}

\newcommand{\nkor}[3]                   
   {\refstepcounter{NCor}\label{#1}
    \hypertarget{#1}{}
    \begin{normalblock}                         
       {Korollar \theNCor: #2.}
       #3
    \end{normalblock}}
\newcommand{\ncor}[3]                   
   {\refstepcounter{NCor}\label{#1}
    \hypertarget{#1}{}
    \begin{normalblock}                         
       {Corollary \theNCor: #2.}
       #3
    \end{normalblock}}
}

\mode<presentation>{    
\newcounter{NCor}   
\def\theNCor{\arabic{NCor}}   
\newcommand{\nkor}[3]
   {\only<1>{\refstepcounter{NCor}}\label{#1} 
    \begin{normalblock}
       {Korollar: #2.}                    % \theNCor
       #3
    \end{normalblock}}
\newcommand{\ncor}[3]
   {\only<1>{\refstepcounter{NCor}}\label{#1} 
    \begin{normalblock}
       {Corollary: #2.}                   % \theNCor
       #3
    \end{normalblock}}
}


% Beispiel // Example
\mode<article>{ 
\newcounter{NEx}[section]
\def\theNEx{\thesection.\arabic{NEx}}

\newcommand{\ebsp}[3]                   
   {\refstepcounter{NEx}\label{#1}
    \hypertarget{#1}{}
    \begin{exampleblock}                         
       {Beispiel \theNEx: #2.}
       #3
    \end{exampleblock}}
\newcommand{\eex}[3]                   
   {\refstepcounter{NEx}\label{#1}
    \hypertarget{#1}{}
    \begin{exampleblock}                         
       {Example \theNEx: #2.}
       #3
    \end{exampleblock}}
}

\mode<presentation>{    
\newcounter{NEx}   
\def\theNEx{\arabic{NEx}}   
\newcommand{\ebsp}[3]
   {\only<1>{\refstepcounter{NEx}}\label{#1} 
    \begin{exampleblock}
       {Beispiel: #2.}                    % \theNEx
       #3
    \end{exampleblock}}
\newcommand{\eex}[3]
   {\only<1>{\refstepcounter{NEx}}\label{#1} 
    \begin{exampleblock}
       {Example: #2.}                     % \theNEx
       #3
    \end{exampleblock}}
}


%-----------------------------------------------------------------------------%
%                          10. Block & Box Shortcuts                          %
%-----------------------------------------------------------------------------%

% nblock{Title}{Body}                       BLUE
\newcommand{\nblock}[2]
		   {\begin{normalblock}{#1}
			   #2
			\end{normalblock}}

% nbox{Body}
\newcommand{\nbox}[1]
		   {\begin{normalbox}
			   #1
			\end{normalbox}}

% eblock{Title}{Body}                       GREEN
\newcommand{\eblock}[2]
		   {\begin{exampleblock}{#1}
			   #2
			\end{exampleblock}}

% ebox{Body}
\newcommand{\ebox}[1]
		   {\begin{examplebox}
			   #1
			\end{examplebox}}
			
% ablock{Title}{Body}                       RED
\newcommand{\ablock}[2]
		   {\begin{alertblock}{#1}
			   #2
			\end{alertblock}}

% abox{Body}                              
\newcommand{\abox}[1]
		   {\begin{alertbox}
			   #1
			\end{alertbox}}

% tblock{Title}{Body}                       THEME/TEXT
\newcommand{\tblock}[2]
		   {\begin{block}{#1}
			   #2
			\end{block}}

% tbox{Body}
\newcommand{\tbox}[1]
		   {\begin{bodybox}
			   #1
			\end{bodybox}}
	
		
%-----------------------------------------------------------------------------%
%                      11. Boxes in Itemize Environments                      %
%-----------------------------------------------------------------------------%

% inbox{body}
\newcommand{\inbox}[1] 
   {\switch{\begin{minipage}[c]{0.012\columnwidth}
 	        	\strut
     		\end{minipage}%
	 		\begin{minipage}[c]{0.92\columnwidth}
 		    	\begin{normalbox}
			   		#1
				\end{normalbox}
     		\end{minipage}}
     	   {#1}} 

% iebox{body}
\newcommand{\iebox}[1] 
   {\switch{\begin{minipage}[c]{0.012\columnwidth}
 	        	\strut
     		\end{minipage}%
	 		\begin{minipage}[c]{0.92\columnwidth}
 		    	\begin{examplebox}
			   		#1
				\end{examplebox}
     		\end{minipage}}
     	   {#1}} 
     	   
% ieblock{body}
\newcommand{\ieblock}[2] 
   {\switch{\begin{minipage}[c]{0.012\columnwidth}
 	        	\strut
     		\end{minipage}%
	 		\begin{minipage}[c]{0.92\columnwidth}
 		    	\begin{exampleblock}{#1}
			       #2
      			\end{exampleblock}
     		\end{minipage}}
     	   {#1}} 
     	        
% iabox{body}
\newcommand{\iabox}[1] 
   {\switch{\begin{minipage}[c]{0.012\columnwidth}
 	        	\strut
     		\end{minipage}%
	 		\begin{minipage}[c]{0.92\columnwidth}
 		    	\begin{alertbox}
			   		#1
				\end{alertbox}
     		\end{minipage}}
     	   {#1}} 		   
     	   
% itbox{body}
\newcommand{\itbox}[1] 
   {\switch{\begin{minipage}[c]{0.012\columnwidth}
 	        	\strut
     		\end{minipage}%
	 		\begin{minipage}[c]{0.92\columnwidth}
 		    	\begin{bodybox}
			   		#1
				\end{bodybox}
     		\end{minipage}}
     	   {#1}} 
     
% iinbox{body}
\newcommand{\iinbox}[1] 
   {\switch{\begin{minipage}[c]{0.012\columnwidth}
 	        	\strut
     		\end{minipage}%
	 		\begin{minipage}[c]{0.855\columnwidth}
 		    	\begin{normalbox}
			   		#1
				\end{normalbox}
     		\end{minipage}}
     	   {#1}} % iebox{body}

% iiebox{body}
\newcommand{\iiebox}[1] 
   {\switch{\begin{minipage}[c]{0.012\columnwidth}
 	        	\strut
     		\end{minipage}%
	 		\begin{minipage}[c]{0.855\columnwidth}
 		    	\begin{examplebox}
			   		#1
				\end{examplebox}
     		\end{minipage}}
     	   {#1}} 

% iiabox{body}
\newcommand{\iiabox}[1] 
   {\switch{\begin{minipage}[c]{0.012\columnwidth}
 	        	\strut
     		\end{minipage}%
	 		\begin{minipage}[c]{0.855\columnwidth}
 		    	\begin{alertbox}
			   		#1
				\end{alertbox}
     		\end{minipage}}
     	   {#1}} 

% iitbox{body}
\newcommand{\iitbox}[1] 
   {\switch{\begin{minipage}[c]{0.012\columnwidth}
 	        	\strut
     		\end{minipage}%
	 		\begin{minipage}[c]{0.855\columnwidth}
 		    	\begin{bodybox}
			   		#1
				\end{bodybox}
     		\end{minipage}}
     	   {#1}} 


%-----------------------------------------------------------------------------%
%                         12. Listings Configuration                          %
%-----------------------------------------------------------------------------%

% The following requires the listings package to be loaded prior to mybeamer!

\newcommand{\setlistingsnormal}
           {\lstset{language=Matlab,
	         basicstyle=\normalsize\ttfamily,       % normalsize
	         keywordstyle=\color{myblue}\bfseries,
	         identifierstyle=,                      % nothing happens
	         commentstyle=\color{mygreen}, 
	         stringstyle=\color{mymagenta},
	         showstringspaces=false,                % no special string spaces
	         numbers=left, 
	         numberstyle=\tiny, 
	         stepnumber=1, 
	         numbersep=8pt,
            morekeywords={ones,hilbert}}}

\newcommand{\setlistings}
           {\lstset{language=Matlab,
	         basicstyle=\footnotesize\ttfamily,     % print whole listing small
	         keywordstyle=\color{myblue}\bfseries,
	         identifierstyle=,                      % nothing happens
	         commentstyle=\color{mygreen}, 
	         stringstyle=\color{mymagenta},
	         showstringspaces=false,                % no special string spaces
	         numbers=left, 
	         numberstyle=\tiny, 
	         stepnumber=1, 
	         numbersep=8pt,
            morekeywords={ones,hilbert}}}

\newcommand{\setlistingsscript}
           {\lstset{language=Matlab,
	         basicstyle=\footnotesize\ttfamily,     % footnotesize
	         keywordstyle=\color{myblue}\bfseries,
	         identifierstyle=,                      % nothing happens
	         commentstyle=\color{mygreen}, 
	         stringstyle=\color{mymagenta},
	         showstringspaces=false,                % no special string spaces
	         numbers=left, 
	         numberstyle=\tiny, 
	         stepnumber=1, 
	         numbersep=8pt,
	         backgroundcolor=\color{ultralightgray},
	         morekeywords={ones,hilbert}}}

\newcommand{\setlistingsscriptsize}
           {\lstset{language=Matlab,
	         basicstyle=\scriptsize\ttfamily,            % small
	         keywordstyle=\color{myblue}\bfseries,
	         identifierstyle=,                      % nothing happens
	         commentstyle=\color{mygreen}, 
	         stringstyle=\color{mymagenta},
	         showstringspaces=false,                % no special string spaces
	         numbers=left, 
	         numberstyle=\tiny, 
	         stepnumber=1, 
	         numbersep=8pt,
	         morekeywords={ones,hilbert}}}

\newcommand{\setlistingssmall}
           {\lstset{language=Matlab,
	         basicstyle=\scriptsize\ttfamily,       % scriptsize
	         keywordstyle=\color{myblue}\bfseries,
	         identifierstyle=,                      % nothing happens
	         commentstyle=\color{mygreen}, 
	         stringstyle=\color{mymagenta},
	         showstringspaces=false,                % no special string spaces
	         numbers=left, 
	         numberstyle=\tiny, 
	         stepnumber=1, 
	         numbersep=8pt,
            morekeywords={ones,hilbert}}}

\newcommand{\setlistingstiny}
           {\lstset{language=Matlab,
	         basicstyle=\tiny\ttfamily,             % tiny
	         keywordstyle=\color{myblue}\bfseries,
	         identifierstyle=,                      % nothing happens
	         commentstyle=\color{mygreen}, 
	         stringstyle=\color{mymagenta},
	         showstringspaces=false,                % no special string spaces
	         numbers=left, 
	         numberstyle=\tiny, 
	         stepnumber=1, 
	         numbersep=8pt,
            morekeywords={ones,hilbert}}}

% Nonumber	         
\newcommand{\setlistingsnormalnonumber}
           {\lstset{language=Matlab,
	         basicstyle=\normalsize\ttfamily,       % normalsize
	         keywordstyle=\color{myblue}\bfseries,
	         identifierstyle=,                      % nothing happens
	         commentstyle=\color{mygreen}, 
	         stringstyle=\color{mymagenta},
	         showstringspaces=false,                % no special string spaces
	         numbers=none, 
	         numberstyle=\tiny, 
	         stepnumber=1, 
	         numbersep=8pt,
	         morekeywords={ones,hilbert}}}
	         
\newcommand{\setlistingsnonumber}
           {\lstset{language=Matlab,
	         basicstyle=\footnotesize\ttfamily,     % footnotesize
	         keywordstyle=\color{myblue}\bfseries,
	         identifierstyle=,                      % nothing happens
	         commentstyle=\color{mygreen}, 
	         stringstyle=\color{mymagenta},
	         showstringspaces=false,                % no special string spaces
	         numbers=none, 
	         numberstyle=\tiny, 
	         stepnumber=1, 
	         numbersep=8pt,
	         morekeywords={ones,hilbert}}}	         

\newcommand{\setlistingssmallnonumber}
           {\lstset{language=Matlab,
	         basicstyle=\small\ttfamily,            % small
	         keywordstyle=\color{myblue}\bfseries,
	         identifierstyle=,                      % nothing happens
	         commentstyle=\color{mygreen}, 
	         stringstyle=\color{mymagenta},
	         showstringspaces=false,                % no special string spaces
	         numbers=none, 
	         numberstyle=\tiny, 
	         stepnumber=1, 
	         numbersep=8pt,
	         morekeywords={ones,hilbert}}}	         

\newcommand{\setlistingstinynonumber}
           {\lstset{language=Matlab,
	         basicstyle=\tiny\ttfamily,             % tiny
	         keywordstyle=\color{myblue}\bfseries,
	         identifierstyle=,                      % nothing happens
	         commentstyle=\color{mygreen}, 
	         stringstyle=\color{mymagenta},
	         showstringspaces=false,                % no special string spaces
	         numbers=none, 
	         numberstyle=\tiny, 
	         stepnumber=1, 
	         numbersep=8pt,
	         morekeywords={ones,hilbert}}}	         

% Commandwindow, all B/W
\newcommand{\setlistingscommandwindow}
           {\lstset{language=Matlab,
	         basicstyle=\footnotesize\ttfamily,     % footnotesize
	         keywordstyle=\color{black},
	         identifierstyle=,                      % nothing happens
	         commentstyle=\color{black}, 
	         stringstyle=\color{black},
	         showstringspaces=false,                % no special string spaces
	         numbers=none, 
	         numberstyle=\tiny, 
	         stepnumber=1, 
	         numbersep=8pt,
	         frame=lines,
	         morekeywords={ones,hilbert}}}	         

\newcommand{\setlistingscommandwindowsmall}
           {\lstset{language=Matlab,
	         basicstyle=\small\ttfamily,            % small
	         keywordstyle=\color{black},
	         identifierstyle=,                      % nothing happens
	         commentstyle=\color{black}, 
	         stringstyle=\color{black},
	         showstringspaces=false,                % no special string spaces
	         numbers=none, 
	         numberstyle=\tiny, 
	         stepnumber=1, 
	         numbersep=8pt,
	         frame=lines,
	         morekeywords={ones,hilbert}}}     

% Commandwindow, Color
\newcommand{\setlistingscommandwindowcolor}
           {\lstset{language=Matlab,
	         basicstyle=\footnotesize\ttfamily,     % footnotesize
	         keywordstyle=\color{myblue}\bfseries,
	         identifierstyle=,                      % nothing happens
	         commentstyle=\color{mygreen}, 
	         stringstyle=\color{mymagenta},
	         showstringspaces=false,                % no special string spaces
	         numbers=none, 
	         numberstyle=\tiny, 
	         stepnumber=1, 
	         numbersep=8pt,
	         frame=lines,
	         morekeywords={ones,hilbert}}}	         

\newcommand{\setlistingscommandwindowcolorsmall}
           {\lstset{language=Matlab,
	         basicstyle=\small\ttfamily,            % small
	         keywordstyle=\color{myblue}\bfseries,
	         identifierstyle=,                      % nothing happens
	         commentstyle=\color{mygreen}, 
	         stringstyle=\color{mymagenta},
	         showstringspaces=false,                % no special string spaces
	         numbers=none, 
	         numberstyle=\tiny, 
	         stepnumber=1, 
	         numbersep=8pt,
	         frame=lines,
	         morekeywords={ones,hilbert}}}

	         
%-----------------------------------------------------------------------------%
%        13. Textpos Settings: Place Text & Boxes by Absolute Position        %
%                                                                             %
%   Requires \usepackage[overlay,absolute]{textpos}                           %
%            \usepackage{rotating}                                            %
%                                                                             %
%-----------------------------------------------------------------------------%

% Usage
% \sumbox{5}{5}{5}{5}{\bc\huge\redb{IMPORTANT!}\ec}
% \sumboxframe{c}{\frametitle{Test}xxx}
% {5}{5}{5}{5}{\bc\huge\redb{IMPORTANT!}\ec}


% Set Grid
\TPGrid{10}{10} % Number of grid units per paper width/height

\setbeamercolor{postit}{fg=black,bg=yellow}

% \sumbox
% #1: Textwidth
% #2: Horizontal position (0 ? 10)
% #3: Vertical position   (0 ? 10)
% #4: Rotation angle (anti-clockwise)
% #5: Text
\newcommand{\sumbox}[5]{
   \begin{textblock}{#1}[0.5,0.5](#2,#3){ 
       \begin{turn}{#4}%
       \begin{beamerboxesrounded}[upper=postit,lower=postit,shadow=true]
       
           #5
       \end{beamerboxesrounded}
       \end{turn}}
   \end{textblock}}

% \posbox
% #1: Textwidth
% #2: Horizontal position (0 ? 10)
% #3: Vertical position   (0 ? 10)
% #4: Rotation angle (anti-clockwise)
% #5: Text
\newcommand{\posbox}[5]{
   \begin{textblock}{#1}[0.5,0.5](#2,#3){ 
       \begin{turn}{#4}%
           #5
       \end{turn}}
   \end{textblock}}
      
% \sumboxframe
% #1: frame
% #2: Textwidth
% #3: Horizontal position (0 ? 10)
% #4: Vertical position   (0 ? 10)
% #5: Rotation angle (anti-clockwise)
% #6: Text
\newcounter{saveframenumber}
\newcommand{\sumboxframe}[7]{
   \begin{frame}[#1] 
      #2 
      \setcounter{saveframenumber}{\value{framenumber}}
   \end{frame}
   \begin{frame}[#1] 
      \setcounter{framenumber}{\value{saveframenumber}}
      #2 
      \begin{textblock}{#3}[0.5,0.5](#4,#5){ 
         \begin{turn}{#6}%
         \begin{beamerboxesrounded}[upper=postit,lower=postit,shadow=true]
       
            #7
         \end{beamerboxesrounded}
         \end{turn}}
       \end{textblock}
   \end{frame}}

	                 
%------------------------------------EOF--------------------------------------%
